\documentclass[12pt]{report}
\usepackage[T2A]{fontenc}
\usepackage[utf8]{inputenc}
\usepackage[english,russian]{babel}
\usepackage{listings}

\renewcommand\contentsname{Оглавление}

\usepackage{geometry}
\geometry{left=2cm}
\geometry{right=1.5cm}
\geometry{top=1cm}
\geometry{bottom=2cm}

% Для листинга кода:
\lstset{ %
language=python,                 % выбор языка для подсветки (здесь это С)
basicstyle=\small\sffamily, % размер и начертание шрифта для подсветки кода
numbers=left,               % где поставить нумерацию строк (слева\справа)
numberstyle=\tiny,           % размер шрифта для номеров строк
stepnumber=1,                   % размер шага между двумя номерами строк
numbersep=5pt,                % как далеко отстоят номера строк от подсвечиваемого кода
showspaces=false,            % показывать или нет пробелы специальными отступами
showstringspaces=false,      % показывать или нет пробелы в строках
showtabs=false,             % показывать или нет табуляцию в строках
frame=single,              % рисовать рамку вокруг кода
tabsize=2,                 % размер табуляции по умолчанию равен 2 пробелам
captionpos=t,              % позиция заголовка вверху [t] или внизу [b] 
breaklines=true,           % автоматически переносить строки (да\нет)
breakatwhitespace=false, % переносить строки только если есть пробел
escapeinside={\#*}{*)}   % если нужно добавить комментарии в коде
}

\usepackage{titlesec} % подключаем нужные пакеты
\newcommand{\hsp}{\hspace{20pt}} % длина линии в 20pt
\titleformat{\chapter}[hang]{\Huge\bfseries}{\thechapter{.}\hsp}{0pt}{\Huge\bfseries}

\begin{document}

    \begin{titlepage}

        \begin{center}
            \large
            {\slГосударственное образовательное учреждение высшего профессионального образования\\
            {\bf«Московский государственный технический университет имени Н. Э. Баумана»\\
				(МГТУ им. Н.Э. Баумана)}}
				\noindent\rule{\textwidth}{2pt}
		\end{center}
				{\large ФАКУЛЬТЕТ\hsp«Информатика и система управления»}\\
				{\large КАФЕДРА\hsp«Программное обеспечение ЭВМ и информационные технологии»}\\
		\begin{center}
            \vspace{3cm}

				{\scshape\large Лабораторная работа №1 \par}
				\vspace{0.5cm}	
				{\scshape\large по курсу "Анализ алгоритмов" \par}
				\vspace{1.5cm}
				{\huge\bfseries Расстояние Левенштейна и Дамерау-Левенштейна \par}
				\vspace{2cm}
				\large Выполнил: Сорокин Антон, гр. ИУ7-52Б\\
				\vspace{0.5cm}
				{\largeПреподаватели: Волкова Л.Л., Строганов Ю.В.}
			
				\vfill
				\large \textit {Москва, 2019г.}
            
            \end{center}

    \end{titlepage}
	
	\tableofcontents

	\chapter*{Введение}
	\addcontentsline{toc}{chapter}{Введение}

	{\bf Расстояние Левенштейна} определяет минимальное количество операций, необходимых для превращения одной строки в другую. Задача определения такого минимума актуальна, так как она решает множество проблем в теории информации и компьютерной лингвистике, например:

	\begin{itemize}
		\item исправление ошибок в словах при вводе (при в поисковых ситсемах, базах данных, программах автоматического определения текста);
		\item сравнении текстовых файлов (к примеру, утилита diff);
		\item сравнение белков, генов и хромосом в биоинформатике.
	\end{itemize}

    \chapter{Аналитическая часть}

	 \section{Задачи}
	Цель лабораторной работы: исследовать расстояния Левенштейна и Дамерау-Левенштейна. Для достижения этой цели были поставлены следующие задачи: 
	\begin{itemize}
		\item изучить алгоритмы вычисления расстояний между строками;
		\item применить методы динамического программирования для матричной реализации алгоритмов;
		\item сравнить матричную и рекурсивную реализацию алгоритмов;
		\item оценить эффективность каждой из реализаций по времени и памяти.
	\end{itemize}

	 \section{Описание алгоритмов}
    При нахождении расстояния Левенштейна определяется минимального количество операций следующих видов:
	\begin{itemize}
		\item вставка (I - insert);
		\item удаление (D - delete);
		\item замена (R - replace);
		\item совпадение (M - match).	
	\end{itemize}
	При нахождении расстояния Дамерау-Левенштейна добавляется операция транспозиции (T - transpose), или перестановки двух соседних символов.\\

	Таким образом, если заданы две строки $S_{1}$ и $S_{2}$ с длинами $m$ и $n$ соответственно над некоторым алфавитом, то расстояние Левенштейна можно вычислить по следующей рекуррентной формуле:

	\begin{equation}
	D(S_{1}[1..m], S_{2}[1..n]) = min(D(S_{1}[1..m-1], S_{2}[1..n] + 1),\\
	\end{equation}
	\begin{equation}
	D(S_{1}[1..m], S_{2}[1..n-1]+1),\\
	\end{equation}
	\begin{equation}
	D(S_{1}[1..m-1], S_{2}[1..n-1]+m(S_{1}[m], S_{2}[n])))
	\end{equation}
	где
	\begin{displaymath}
	m(a,b) = \left\{
	\begin{array}{ll}
		0, a=b\\
		1, a \neq b
	\end{array} \right.,
	\end{displaymath}\ \\
	Соотношения в рекурретной формуле отвечают за соотвествующие разрешённые операции:
	\begin{itemize}
		\item (1.1) - вставка;
		\item (1.2) - удаление;
		\item (1.3) - замена или совпадение в зависимости от результата $m(a,b)$.	
	\end{itemize}

	При вычислении расстояния Дамерау-Левенштейна в рекурретную формулу вносится дополнительное соотношение в минимум:
	\begin{equation}
	\label{damerau}
	D(S_{1}[1..m-2], S_{2}[1..n-2])+1
	\end{equation}
Соотношение ~(\ref{damerau}) вносится в качестве дополнительного аргумента минимума только при выполнении следующих условий:
	\begin{itemize}
		\item $m > 2,n > 2$;
		\item $S_{1}[m] = S_{2}[n-1]$;
		\item $S_{1}[m-1] = S_{2}[n]$.	
	\end{itemize}

	Тривиальным случаем в рекуррентной формуле является случай, когда одна из строк пустая. В этом случае расстояние Левенштейна равно длине другой строки.

	Расстояния Левенштейна и Дамерау-Левенштейна можно также вычислить, используя матрицу, в которое разрешённые операции определены следующим образом:
	\begin{itemize}
		\item движение по столбцам вправо - вставка;
		\item движение по строкам вверх - удаление;
		\item движение по диагонали - замена/совпадение.
	\end{itemize}

	\chapter{Конструкторская часть}

	\section{Схемы алгоритмов}

	\section{Структуры данных}
	
	\chapter{Технологическая часть}
	\section{Требования к программному обеспечению}
	\section{Средства реализации}
   \section{Листинг кода}

	\chapter{Экспериментальная часть}
	\section{Примеры работы}
	\section{Тесты}
   \section{Исследование времени выполнения}

   \chapter*{Заключение}
	\addcontentsline{toc}{chapter}{Заключение}

	\chapter*{Список литературы}
	\addcontentsline{toc}{chapter}{Список литературы}

\end{document}
